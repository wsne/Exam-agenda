\chapter{Q10}
\emph{Q10: What are the benefits of data aggregation? What are the challenges
of data aggregation in sensor networks?}

\section{Data-aggregation}
Save energy and increase network lifetime by combining multiple sensor data.
\\
Take advantage of the routing hierarchy and high network density.

\subsection{Data Storage Representations}
Store individual data
\\
Store only mean and variance

\subsection{Aggregation Functions}
Average, max, min, median.
Suppression of duplicates.
\\
Exploit spatial and temporal correlation.

\subsection{Challenges}
Find the optimal number of aggregation points
\\
Selection of aggregation points
\\
Dynamic change of aggregation points (energy efficiency

\subsection{Pros}
Energy and memory efficient
\\
Reduce traffic load
\\
Network congestion
\\
Scalable to arge numbers of sinks and motes
\\
Sending is expensive - computation is cheep
\\
-To send 1kB 100 m uses the same energy as doing 3 million calculations.
\\
Save energy and increase network lifetime

\subsection{Trade offs}

Accuracy 
\\
Making robust estimations
\\
single outlier reading can damage MAX/MIN aggregates

\subsection{Algorithms}
Optimal Data Aggregation is NP-Hard

Sub-optimal Algorithms:
\begin{description}
	\item[Opportunistic] Just aggregate when possible
	\item[Center at Nearest Source (CNS)] The nearest source acts as the
		aggregation point
	\item[Shortest Paths Tree (SPT)] Sources send using shortest path if able
		to aggregate
	\item[Greedy Incremental Tree (GIT)] Recursively select the closest source
		to the tree
	\item[Clustered Diffusion with Dynamic Data Aggregation] Hybrid between
		diffusion and clustering with the ability to aggregate data at the
		cluster heads
\end{description}

